\documentclass{article}
\usepackage[utf8]{inputenc}

\title{Code of Conduct}
\author{Group 18b}
\date{November 2022}

\begin{document}

\maketitle

\section{Meetings}
    \begin{itemize}
        \item Agenda for the meeting is to be submitted by 18:00 on the day before the upcoming meeting; it should describe a clear outline of the discussed items, which on their own outline the goal of the meeting.
        \item Secretary of the team takes notes and publishes those in the repository
        \item Lab meetings have mandatory presence
        \item For other meetings, absence ought to be discussed beforehand
    \end{itemize}
\section{Product}
    \begin{itemize}
        \item  We are to develop a prototype of the application in the provided time window
        \item User story assignment using the MOSCOW principle; Must-have and a part of should-have features will be implemented. Could-have features will not be addressed.
        \item Development and Main branches - (if issues relate to documentation, they ought to be merged directly to main rather than dev if separate dev and main branches are agreed upon)
        \item Consistent code style, possibly enforced via Checkstyle, adjusted during the coding process
        \item Naming convention for issue branches - issue number followed by a short description of the issue in kebab case with lowercase letters
        \item Don't rename other people's variables and classes without consent
        \item Group Grade Aim: 9 or higher
    \end{itemize}
\section{Planning}
    \begin{itemize}
        \item Recurring meetings on Mondays from 11:00 until 12:45
        \item Meetings with our TA are on Thursdays unless specified otherwise
        \item Communication - Usually conducted via Discord, Whatsapp for urgent matters, Mattermost for communication with our TA
        \item Chairmen and Notetakers are rotated on a weekl basis - previous Notetaker becomes new Chairman and new Notetaker either volunteers or is chosen by chairman
    \end{itemize}
\section{(Team)work}
    \begin{itemize}
        \item The CoC can be updated if the entire team agrees with the amendment
        \item Time tracking in Gitlab is encouraged
        \item Templates for agendas for meetings defined in first meeting
        \item When the agendas are published on GitLab, initially they are just named 'Agenda\_DD\_MM\_YYYY' and is later appended with the notes to \newline 'Agenda\_Notes\_DD\_MM\_YYYY'
    \end{itemize}
\section{Code Contribution}
    \begin{itemize}
        \item Each merge request is approved by 2 others ( check if the code is meaningful etc. )
        \item Every merge request should be reviewed by at least two people at latest 24 hours after the creation of the request.
        \item Every merge request should be reviewed by at least two people at latest 24 hours after the creation of the request.
        \item Test coverage for each class above 60 $\%$
        \item Every team member should finish up their weekly Merge Requests by Sunday, 23:59 and the actual Merge Day is Monday to resolve code conflicts, refactorings, etc.
        \item Close to merge day, finishing existing MR's should be prioritized over starting new features.
        \item Reassignment of issues is allowed in this case
        \item Constant meetings and project reviews to find out what needs improvement
        \item Accordingly, whenever a merge request is made, the one making it should inform the team by channels outside of Gitlab, such as Discord or Mattermost
        \item If the merge request is not approved, the one blocking it should explain it
        \item Merge request should not be approved without any feedback. In the case when there are no complaints or questions about a person's code, the reviewer should specify that in their comment
        \item The one making the merge request should put a short description of all the changes in their merge request. If there are stuff still to be added at a later date, the reviewer should describe what with "TO-DO" with a time limit they set for doing so with another merge request.
        \item Merge request reviews should be done in the style of conventional comments: conventionalcomments.org/
        \item When addressing TO-DO parts of a previous merge request in another, they should refer to that merge request explicitly in their comments
        \item If there is a known bug in the merge request, which is to be fixed, it should be denoted with the "BUG" label in a new issue and fixed as fast as possible. It should additionally be avoided to submit merge request with bugs unless the one submitting it feels it is absolutely necessary
        \item All commits should be atomic
        \item Avoid commits like "fixed pipeline" and "bugfix" by squashing it into the appropriate commit.
        \item You should do this before pushing:
            \begin{itemize}
                \item git rebase -i HEAD~[number of commits to go back]
                \item If already pushed, git push --force for new history
            \end{itemize}
        \item For every commit, the pipeline should pass
        \item One feature per commit
        \item The issues that reviewers mention in their comments should be at least addressed in 12 hours from the time of the comment (excluding the hours between 22:00 and 09:00 the next day to account for sleep )
        \item Never modify others' branches without consent
    \end{itemize}
\section{Conflicts}
    \begin{itemize}
        \item Reaching consensus first, otherwise - majority vote, the chair is the tiebreaker
        \item In case of a group member not adhering by the CoC first try to address the issue within our group, but if that does not yield results, address higher authority without delays
    \end{itemize}
\end{document}
